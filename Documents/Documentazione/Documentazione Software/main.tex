\documentclass[a4paper,11pt]{article}       % Articolo, Foglio A4, Carattere 11
\usepackage[T1]{fontenc}                    % Codifica dei font
\usepackage[utf8]{inputenc}                 % Codifica dell'input da tastiera
\usepackage[italian]{babel}                 % Lingua: italiano
\usepackage{hyperref}                       % Collegamenti ipertestuali  
\usepackage[toc]{glossaries}                     % Gestione glossario

% Dichiarazione glossario
\makeglossaries
\newglossaryentry{database}
{
    name = database,
    description = {Archivio di dati strutturato in modo da razionalizzare la gestione e l'aggiornamento delle informazioni e da permettere lo svolgimento di ricerche complesse}
}

\newglossaryentry{XML}
{
    name = XML,
    description = {Sigla di eXtensible Markup Language, è un metalinguaggio per la definizione di linguaggi di markup, ovvero un linguaggio marcatore basato su un meccanismo sintattico che consente di definire e controllare il significato degli elementi contenuti in un documento o in un testo}
}


\newglossaryentry{sistema operativo}
{
    name = sistema operativo,
    description = {Piattaforma che gestisce le risorse hardware e software della macchina}
}

\newglossaryentry{server FTP}
{
    name = server FTP,
    description = {Componente di elaborazione con lo scopo di gestire il trasferimento dei file}
}

%-----------------------------------------------------------------


\begin{document}
\begin{titlepage}
    \centering
        \vspace*{1in}
        \begin{Large}
            Progettazione di software gestionale
  
            per bar da mimmo\par
        \end{Large} 
        \vspace{1.5in}
        \vfill
        Documento dei requisiti\\
        \vspace{0.1in}ver. 1.0
        \par
        \vspace{0.5in}
            \date{05/12/2020}
        \par
\end{titlepage}

% numerazione romana delle pagine
\pagenumbering{roman}

%Indice dei contenuti
\tableofcontents
\newpage

%Numerazione araba delle pagine
\pagenumbering{arabic}

\section{Premesse del progetto}
    \subsection{Obiettivi e scopo del progetto}
    Il prodotto è appositamente studiato per gestire e organizzare gli ordini da parte dei clienti. Il prodotto è realizzato direttamente in base alle richieste degli operatori del settore. Risultano immediate e semplici le attività di tutti i giorni: ordinazione e ricezione degli ordini.
    
    \subsection{Contesto di business}
    A causa del COVID-19 i clienti sono impossibilitati a recarsi al bar.
    
    \subsection{Stakeholders}
    Le figure che influenzano lo sviluppo del sistema software sono:
    \begin{itemize}
        \item proprietario del bar
        \item barista/fattorino
        \item developers (analisti, progettisti, designer)
    \end{itemize}
    
\section{Servizi del sistema}
    \subsection{Requisiti funzionali}
        
        \begin{itemize}
            \item[2.1.1] Il sistema dovrà consentire la modifica delle impostazioni del sistema software stesso.
                \begin{itemize}
                    \item[2.1.1.1] Il sistema dovrà consentire la modifica del \gls{database} dei prodotti.
                    \item[2.1.1.2] Il sistema dovrà consentire la modifica degli orari del bar.
                    \item[2.1.1.3] Il sistema dovrà consentire il download della cronologia delle ordinazioni.
                    \item[2.1.1.4] Il sistema dovrà consentire l'eliminazione della cronologia delle ordinazioni.
                \end{itemize}
            \item[2.1.2] Il sistema dovrà consentire la visualizzazione delle ordinazioni.
            \item[2.1.3] Il sistema dovrà consentire l'evasione delle ordinazioni.
            \item[2.1.4] Il sistema dovrà consentire la scelta dei prodotti in vendita.
        \end{itemize}
        
    \subsection{Requisiti di informativi}
        \begin{itemize}
            \item[2.2.1] Un'ordinazione è strutturata nel seguente modo:
                \begin{itemize}
                    \item cibi
                    \item bevande
                    \item classe di consegna
                    \item lotto e piano di consegna
                    \item orario di consegna
                    \item numero di telefono
                    \item prezzo totale
                \end{itemize}
        \end{itemize}

\section{Vincoli di sistema}

    \subsection{Requisiti di interfaccia}
    L'interfaccia proposta dal programma, sfruttando tutte le funzionalità dell'ambiente operativo 
    a finestre, permette quindi anche all'utente non specializzato di avvicinarsi al programma con facilità.
        \begin{itemize}
            \item[3.1.1] Interfaccia software principale del barista.
                \begin{itemize}
                    \item[3.1.1.1] Visualizzazione del logo del bar.
                    \item[3.1.1.2] Visualizzazione dell'ora corrente.
                    \item[3.1.1.3] Visualizzazione delle ordinazioni in una lista.
                    \item[3.1.1.4] Presenza di un bottone di evasione degli ordini.
                    \item[3.1.1.5] Presenza di un bottone per modificare i prodotti in vendita.
                    \item[3.1.1.6] Presenza di un bottone per accedere alle impostazioni del software.
                \end{itemize}
                
            \item[3.1.2] Interfaccia delle impostazioni del software per il barista.
                \begin{itemize}
                    \item[3.1.2.1] Visualizzazione del logo del bar.
                    \item[3.1.2.2] Visualizzazione dell'ora corrente.
                    \item[3.1.2.3] Presenza di un bottone per modificare il \gls{database} dei prodotti.
                    \item[3.1.2.4] Presenza di un bottone per modificare l'orario del bar.
                    \item[3.1.2.5] Presenza di un bottone per il download della cronologia delle ordinazioni.
                    \item[3.1.2.6] Presenza di un bottone per l'eliminazione della cronologia delle ordinazioni.
                    \item[3.1.2.7] Presenza di un bottone che rimanda all'interfaccia principale.
                \end{itemize}
            
            \item[3.1.3] Interfaccia software del barista per modifica dei prodotti in vendita.
                \begin{itemize}
                    \item[3.1.3.1] Visualizzazione del logo del bar.
                    \item[3.1.3.2] Visualizzazione dell'ora corrente.
                    \item[3.1.3.3] Presenza di una lista contenente tutti gli elementi presenti nel \gls{database} dei prodotti.
                    \item[3.1.3.4] Presenza di una lista contenente i prodotti in vendita.
                    \item[3.1.3.5] Presenza di un bottone di aggiunta dal \gls{database} dei prodotti alla lista dei prodotti in vendita.
                    \item[3.1.3.6] Presenza di un bottone per la rimozione del prodotto selezionato dalla lista dei prodotti in vendita.
                    \item[3.1.3.7] Presenza di un bottone di conferma che riporta all'interfaccia principale salvando le modifiche apportate.
                    \item[3.1.3.8] Presenza di un bottone di annullamento che rimanda all'interfaccia principale senza salvare le modifiche apportate.
                \end{itemize}
                
            \item[3.1.4] Interfaccia software del barista per modificare il \gls{database} dei prodotti.
                \begin{itemize}
                    \item[3.1.4.1] Visualizzazione del logo del bar.
                    \item[3.1.4.2] Visualizzazione dell'ora corrente.
                    \item[3.1.4.3] Presenza di una lista contenente gli elementi presenti nel \gls{database} dei prodotti.
                    \item[3.1.4.4] Presenza di un bottone per rimuovere gli elementi selezionati dalla lista.
                    \item[3.1.4.5] Presenza di un campo di inserimento testuale per il nome del prodotto.
                    \item[3.1.4.6] Presenza di un campo di selezione della tipologia del prodotto (cibo/bevanda).
                    \item[3.1.4.7] Presenza di un selettore per il prezzo del prodotto.
                    \item[3.1.4.8] Presenza di un bottone per aggiungere il nuovo prodotto al \gls{database} dei prodotti.
                    \item[3.1.4.9] Presenza di un bottone che rimanda alle impostazioni.
                \end{itemize}
                
            \item[3.1.5] Interfaccia software del barista per modificare l'orario del bar.
                \begin{itemize}
                    \item[3.1.5.1] Visualizzazione del logo del bar.
                    \item[3.1.5.2] Visualizzazione dell'ora corrente.
                    \item[3.1.5.3] Presenza di un selettore per modificare l'orario di apertura.
                    \item[3.1.5.4] Presenza di un selettore per modificare l'orario di chiusura.
                    \item[3.1.5.5] Presenza di un bottone di conferma che salva le modifiche apportate e rimanda alle impostazioni.
                    \item[3.1.5.6] Presenza di un bottone di annullamento che rimanda alle impostazioni senza salvare le modifiche apportate.
                \end{itemize}
            
        \end{itemize}
        
    \subsection{Requisiti tecnologici}
        \begin{itemize}
            \item[3.2.1] Un file è adibito a \gls{database}.
            \item[3.2.2] Vengono utilizzati diversi file per la comunicazione tra i vari attori.
            \item[3.2.3] Il formato dei file utilizzati è \gls{XML}.
        \end{itemize}
    
    \subsection{Requisiti di prestazione}
    Non si registrano particolari esigenze in questo ambito.
    
    \subsection{Requisiti di sicurezza}
    Non si registrano particolari esigenze in questo ambito.
    
    \subsection{Requisiti operativi}
    L'intero progetto è stato sviluppato utilizzando il linguaggio Java.
	Si relaziona con \gls{sistema operativo} Unix e con \gls{server FTP}.
    
    \subsection{Requisiti politici e legali}
    Non si registrano particolari esigenze in questo ambito.
    
    \subsection{Altri vincoli}
    Questa sezione è vuota.

\clearpage
\section{Appendici}
    \printglossary[nonumberlist]
\end{document}
